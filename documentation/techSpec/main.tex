\documentclass{article}
\usepackage[utf8]{inputenc}

\title{Topic Modeling Library and Framework Specification}
\author{Valeriy Avanesov, Ilya Kozlov}
\date{February 2014}

\begin{document}

\maketitle

\section{Terminology}
    \begin{itemize}
        \item Textual document \--- Textual document is a map attribute $\to$ sequence of words. 
        \item Document \--- Document is a map from attribute to sequence of integer numbers.
        \item Topic \--- multinomial distribution on words. 
    \end{itemize}

\section{Introduction}
    \subsection{Generative model}
        In topic modeling every document viewed as mixture of topics. Each topic is a multinomial distribution on words, 
        so generation model may be defined as follow:
        \begin{itemize} \label{generation}
            \item For every position in document $d$ i.i.d choose topic $t$ from distribution of topics by document
            \item Choose word $w$ from topic $t$
        \end{itemize}
        The aim of topic modeling is to recover topics and distribution of document by topics. 
        \subsubsection{Polylingual topic model}
	    Suppose one has a collection of documents on different languages. We have a prior knowledge that some of the
	    documents (or all of them) are written on the same topic but in different languages (for example one document may be a translation of another).
	    Wikipedia may be considered as a source of this type of data. It leads us to a polylingual topic modeling \cite{polylingual}.
	    In this model we assume that every topic is a set of multinomial distributions, one per language. Also we assume that
	    every document may hold more than one set of words, so we represent document as map attribute $\to$ set of words.  

        \subsubsection{Robust PLSA}
            In robust PLSA we assume that some of words too specific for document and may not be explained by topic 
            distribution. Conversely, some of words too common and may be explained by any topic.
            Robust PLSA take it into account. In robust PLSA word may be generated from topic, noise or background.
            Noise is a multinomial distribution on rare words. Every document has its own noise.
            Background is a multinomial distribution on common words. Background is one for all documents.  
            To generate words $w$  in document $d$ we: 
            \begin{itemize}
                \item with probability $\gamma$ generate word from noise
                \item with probability $\varepsilon$ generate word from background
                \item with probability $1 - \gamma - \varepsilon$ generate word from topic
		    distribution as in \ref{generation}
            \end{itemize}
            $\gamma$ and $\varepsilon$ are hyperparameter of model.  
        
        \subsubsection{Sparse PLSA}
            One document often corresponds to only a few topics, not to every. Analogously,
            word may corresponds only to a few topics, not to every topic. Sparse PLSA takes it into account,
            and replace some of topics weights in document by zero and replace some weights in distribution
            of words by topic by zero. Sparsification of topic modeling has some features, which allow to sparse distribution of 
            words by topic and distribution of document by topics without decreasing of model quality. 
    
    
        \subsection{Topic modeling as optimization problem}
            According to generative model one can estimate probability to observe collection $D$ as:
            \begin{equation} p(D) = \prod_{d \in D} \prod_{w \in d} \sum_{t} p(t|d) p(w|t) \end{equation}
            Denote $\varphi_{wt} = p(w|t)$ and $\theta_{td} = p(t|d)$. One may obtain $\varphi_{wt}$ 
            and $\theta_{td}$ as solution of optimization problem
            
            \begin{equation} \label{optimization} L = \sum_{d \in D} \sum_{w \in d} \log \sum_{t} \varphi_{wt} \theta_{td}  \to \max \end{equation}
	     with boundary
            \begin{equation} \forall t \ \ \sum_{w} \varphi_{wt} = 1, \ \ \forall d \ \ \sum_{w} \theta_{td} = 1 \end{equation}
            and 
            \begin{equation} \forall t, w \ \  \varphi_{wt}  \geq 0, \ \ \forall d, t \ \ \theta_{wt}  \geq 0 \end{equation}
    
        \subsection{Topic modeling as matrix decomposition}
        
            \subsubsection{Kullback-Leibler divergence}
                Kullback-Leibler divergence is a non-negative measure of difference between two different probability distribution:
                \begin{equation} KL(p_i||q_i) = \sum_{i=1}^n p_i \ln\left(\frac{p_i}{q_i}\right)  \end{equation}
                Consider an empirical distribution $\hat{p}_i$ and some parametric distribution $q_i = q_i(\alpha)$ which is used to explain $\hat{p}_i$ . 
                Easy to see that in this case minimization of KL\---divergence is equivalent to estimation of $\alpha$ by maximum-likelihood: 
                \begin{equation} KL(p_i||q_i(\alpha)) = \sum_{i=1}^n p_i \ln\left(\frac{p_i}{q_i(\alpha)}\right) \to \min_{\alpha} 
                \Leftrightarrow \sum_{i=1}^n p_i \ln(q_i(\alpha)) \to \max_{\alpha} \end{equation}
                
                Thus one can easily see that (\ref{optimization}) equivalent to weighted Kullback-Leibler divergence minimization:
                \begin{equation} 
                    \sum_{d \in D} n_d KL_w \left( \frac{n_{dw}}{n_d} || \sum_{t \in T} \varphi_{wt}\theta_{td} \right) \to \min_{\Phi, \Theta}
                \end{equation}
                 where $n_{wd}$\--- number of words $w$ in document $d$, $n_d$ \--- number of words in document $d$.
            
            \subsubsection{Matrix decomposition}
                Denote empirical distribution of words by document as $\hat{p}(w, d) = \frac{n_{wd}}{n_d}$. 
                According to this notation one can consider the problem (\ref{optimization}) as matrix decomposition:
                \begin{equation} F \approx_{KL} \Phi \Theta \end{equation}
                where matrix $F = (\hat{p}(w, d))_{W \times D}$ is empirical distribution of words by document,
                matrix $\Phi = (\varphi_{wt})_{W \times D}$ is distribution of words by topics and 
                matrix  $\Theta = (\theta_{td})_{T\times D}$ is distribution of documents by topics. 
                Thus, our optimization problem may be rewritten in Kullback–Leibler notation as 
                \begin{equation} KL(F , \Phi \Theta) \rightarrow \min \end{equation}
                Thus PLSA may be observed as stochastic matrix decomposition. 
        
        \subsection{Expectation\--Maximization algorithm}
             Unfortunately (\ref{optimization}) has no analytical solution. Thus we use Expectation \-- Maximization (EM) algorithm.
             This algorithm consists of two steps:
                \begin{enumerate}
                    \item Estimation of number $n_{dwt}$ of words $w$, produced by topic $t$ in document $d$. (E \-- step)
                    \item Optimization of distribution of documents by topics and optimization of distribution of topics by words relying on
			the $n_{dwt}$ values obtained during E \-- step . (M \-- step)
                \end{enumerate}  
              One can estimate $n_{dwt}$ as follows:
              \begin{equation}  n_{dwt} = n_{wd} p(w|t) p(t|d) \end{equation} 
              where $n_{wd}$ \--- number of words $w$ in document $d$.
              Thus, probability $p(w|t)$ may be estimated as 
              \begin{equation}  p(w|t) = \frac{n_{wt}}{n_t} = \frac{\sum_d n_{dwt} }{\sum_w \sum_d n_{dwt}}   \end{equation} 
              Similarly for $p(t|d)$
        
        \subsection{Regularizers}
           Regularizers may improve human-understandability of the topics, transform PLSA to LDA, provide an ability 
           for semi-supervised learning (employ a prior knowledge about document topic or topics structure). 
           Instead of optimization (\ref{optimization}) we optimize
            \begin{equation} L(\Phi, \Theta) + R(\Phi, \Theta) \to \max_{\Phi, \Theta} \end{equation}
            Where $R(\Phi, \Theta)$ is a twice differentiable function, named regularizer. 
            Solution of this problem leads to a modification of M\--step:
            \begin{equation} 
                \varphi_{wt} \propto \left(\hat{n}_{wt} + \varphi_{wt} \frac{\partial  R(\Phi, \Theta)}{\partial \varphi_{wt}} \right)_+ ,\ \
                \theta_{td} \propto \left(\hat{n}_{dt} + \theta_{td}\frac{\partial  R(\Phi, \Theta)}{\partial \varphi_{td}} \right)_+
            \end{equation}
            Where $\hat{n}$ has been estimated by E\--step. 
    
    
    \section{Implementation}  
        We take a sequence of documents into input and return distributions of documents by topics and words by topics. We can divide our problem into the following parts: 
        \begin{enumerate}        
            \item Replace every word by its ordered number. For this purpose we use a class Numerator. Then we store a map from words to its ordered number in class Alphabet. 
            \item Choose initial approximation for matrices $\Phi$ and $\Theta$. For this purpose we use class InitialApproxiomationGeneration.
            \item Make E\--step for every attribute in every documents. For this purpose we use class PLSABrick. A single brick processes a single attribute.
                Brick may be robust or non robust or whatever user implements it. 
            \item Then we have to perform M\--step for matrix $\Phi$ (for every attribute) and $\Theta$
        		To perform M\--step for matrix $\Phi$ we use PLSABrick and we in order to perform M\--step for
        		matrix $\Theta$ we use class PLSA.
	        \item After every iteration we check the stop criterion. We use class StoppingCriteria for this purpose.
	        \item In each step we use class Sparcifier in order to make our model sparse. 
        \end{enumerate}
        
        \subsection{Sparcifier}
            We implement standard sparsifier, which replace weight by zero if it less then $threshold$, sparsification starts from 
            iteration $startIteration$ and it does not replace more than $maxNumberOfZeroised$ numbers. $threshold$, $startIteration$ and 
            $maxNumberOfZeroised$ a parameters.
            
                        
        \subsection{Regularizer}
            Every regularizer take into input matrices $\Phi$ and $\Theta$
            We implement the following regularizers: ZeroRegularizer (do nothing), AnticorrelationRegularizer (make topics more different), 
            DirechletRegularizer (transforme PLSA into LDA ). \\
            Regularizer take into input distribution of words by topics (matrix $\Phi$) and distribution of documents by topics (matrix $\Theta$). 
            Regularizer may calculate derivative $\frac{\partial R}{\partial \varphi_{wt}}$ and $\frac{\partial R}{\partial \theta_{wt}}$

    \section{"Quick" start}
        \subsection{Creating a documents}
            Text normalization is not goal of this project. So we suppose that we have sequence of sequence of words (word is a string). 
            First of all we have to create sequence of textual documents. To create TextualDocument type\\
                \texttt{new TextualDocument(Map("lang" -> Seq[String])) }\\
            Then use Numerator \\
                \texttt{ Numerator(Seq[TextualDocument])($Doc_0, Doc_1, Doc_2, \dots Doc_n$) } \\
            This method replace words by its serial number and group words. It also return Alphabet to convert serial number to word. 
        
        \subsection{Creation of PLSA}
            PLSA consist of 4 main parts:
                \begin{enumerate}
                    \item PLSABrick one per attribute. 
                    \item StopingCriteria
                    \item InitiaApproximationGenerator
                    \item ThetaSparsifier
                \end{enumerate}
                
                Construction of PLSABrick is describes in corresponding subsection. 
                One of commonly used stopping criteria is a MaxNumberOfIterationsStopingCriteria (performs fixed number of iteration and terminate)
                    \\ \texttt{val stoppingCriteria = new MaxNumberOfIterationsStopingCriteria(numberOfIterations) } \\ 
                If you don't wont use sparsifier you may use IdenticalSparsifier (it do nothing)
                    \\ \texttt{ val sparsifier = new IdenticalSparsifier() } \\
                Initial approximation generator generate Initial approximation for matrices $\Phi$ and $\Theta$. One of a commonly used method is 
                to draw every position in matrix from uniform distribution and make normalization. For this purposes one may use RandomInitialApproximationGenerator
                    \\ \texttt{ val initialApproximationGenerator  = new RandomInitialApproximationGenerator() } \\
                Now we may construct PLSA:
                    \\ \texttt{ val plsa = new PLSA(brick, sparsifier, initialApproximationGenerator, stoppingCriteria) } \\
            \subsubsection{Creation of PLSABrick}
                PLSABrick consist of three main parts:
                    \begin{enumerate}
                        \item attributeType: type of attribute to process
                        \item phiSparsifier
                        \item regularizer
                    \end{enumerate}
                Thus, before construct the PLSABrick we have to construct phiSparsifier and regularizer. 
                If you don't wont use sparsifier you may use IdenticalSparsifier (it do nothing)\\
                    \texttt{ val sparsifier = new IdenticalSparsifier() } \\
                If you wont to use PLSA without regularizer you may use ZeroRegularizer (it do nothing). To convert PLSA to LDA you may use DirichletRegularizer.
                    \\ \texttt{val regularizer = new ZeroRegularizer()}  for pure PLSA 
                    \\ \texttt{val regularizer = new DirichletRegularizer($\alpha$, $\beta$)}  for LDA.  $\alpha$ and $\beta$ is parameters in dirichlet distribution \\
                
                Now we may construct PLSABrick:
                    \\ \texttt{val brick = new PLSABrick("lang", sparsifier, regularizer) } \\

        \subsection{Using of PLSA}
            Now one may use PLSA for collection of documents:
                \\ \texttt{val trainedModel = plsa.trainOnDocuments(documents) }  \\
                Training model hold distribution of documents by topic (matrix $\Theta$), distribution of words by topics (matrix $\Phi$). Training model can also
                calculate distribution of new document by topics with fixed distribution of words by topic (matrix $\Phi$).
\bibliographystyle{unsrt}
\bibliography{main} 
\end{document} 
