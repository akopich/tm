In this section we will describe how to build a project. 
To build this project you need to have the following program:
\begin{enumerate}
    \item git. It is necessary to obtain source from github. 
	In Ubuntu one can obtain git by typo 
	\begin{lstlisting} 
	    sudo apt-get install git 
	\end{lstlisting}

	
    \item The project is written on Scala, thus you need to have Scala. One can obtain the latest version from \url{http://www.scala-lang.org/}. 
    
    \item Java is also would be useful. One can obtain the latest version from \url{https://www.oracle.com/java/}
    
    \item To build the project you need to have sbt. One can obtain the latest version from \url{http://www.scala-sbt.org/}
\end{enumerate}

Now lets build tm. First of all we have to obtain the source:
\begin{lstlisting}
    git clone https://github.com/ispras/tm.git 
\end{lstlisting}
Go to the directory with tm:
\begin{lstlisting}
    cd tm
\end{lstlisting}
And now we can build project:
\begin{lstlisting}
    sbt assembly
\end{lstlisting}
The .jar file will occur in target/scala\--2.11/ directory. 
Lets run an example task 
\begin{lstlisting}
    java -classpath target/scala-2.11/tm-assembly-1.0.jar ru.ispras.modis.tm.scripts.QuickStart
\end{lstlisting}
It will start topic modeling on the 1000 scientific articles. You can add .jar file in you 
project.

