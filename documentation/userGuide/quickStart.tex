In order to use topic modeling one have to perform the following step:
\begin{enumerate}
    \item Read documents
    \item Split each document into a sequence of words
    \item Replace words by it serial number
    \item Build a topic model.
    \item Train the topic model.
    \item Save results or use it in application. 
\end{enumerate}

In this section would be shown how to perform each step. 
\subsection{Read the data}
    First of all we have to read the data from disc. It worth mentioning that text normalization is non\--goal of our project, thus text should be already preprocessed. 
    This step depends on the organization of input data, so it is
    no use to describe this step in general way. Instead of this we describe this step for our example file, one may easy modify this step for his own use case. % !! криво
    In this introduction we would use file arxiv.part from directory examples. This is a thousand of scientific  papers 
    from Arxiv\footnote{\url{http://arxiv.org/}}. Texts was preprocessed, words were separated by space. Every line corresponds
    to a single document.\\
    First of all one has to read data, and split each line by space:\\
    \texttt{val lines = Source.fromFile(new File("examples/arxiv.part")).getLines()}
    \texttt{val wordsSequence = lines.map(line => line.split(" "))}\\
    Than one have to construct a TextualDocument from each sequence of words:\\
    \texttt{val textualDocuments = wordsSequence.map(words => \\  new TextualDocument(Map(DefaultAttributeType -> words)))}\\
    Each TextualDocument is a map from attribute to the correspond sequence of words. For example attribute may denote language of
    text in the case of multilingual topic modeling. 
    If there is only one attribute (as in our case) one can use \texttt{DefaultAttributeType}.  Furthermore, in such a case one can use \texttt{SingleAttributeNumerator} instead on \texttt{Numerator} -- it takes \texttt{Seq[String]} as inputes and implicitly uses \texttt{DefaultAttributeType}.
\subsection{Replace words by it serial number}
    In order to save the memory in our application we have to replace words by it serial number and group the same words together.
    $$ Seq(duck, \ \ duck, \ \ duck, \ \ goose) \to Seq((0, 3), (1, 1))) $$
    For this purpose we may use object Numerator. It takes into input sequence of textual documents, replace words by it serial number,
    group the same words and return sequence of documents. It also return Alphabet (map form word to serial number and vice versa)
    \texttt{val (documents, alphabet) = Numerator(textualDocuments)}.\\ 
    \subsubsection{How to use alphabet}
	Alphabet holds a map from words to their indices and vice versa. Thus it allow to
	\begin{enumerate}
	    \item get words by it index and attribute:\\
		\texttt{alphabet(Category, 1) // goose}
	    \item get index of word by attribute and word: \\
		\texttt{alphabet.getIndex(Category, duck) // 0}
	    \item get the number of words, corresponding to the given attribute \\
		\texttt{alphabet.numberOfWords(Category) // 100500}
	\end{enumerate}
	
\subsection{Build model}
    One can build model with class Builder (see package ru.ispras.modis.builder). Our project provides three types of builders:
    \begin{enumerate}
	\item PLSABuilder \--- builds a standard PLSA (as it is described in original paper \cite{PLSA_original})
	\item LDABuilder  \--- builds an LDA (PLSA with Dirichlet regularizer, for details section \ref{Regularizers})
	\item RobustPLSABuilder \--- builds a robust PLSA. Robust PLSA takes into account that some words are too rare and can't be explained by any topic 
	    (we call this kind of words "noise"). Some other words may be too common (for example a stop\--word, that we forget to remove in the preprocessing step). Words of this kind
	    are referred to any topic (we call that kind of word "background").  
    \end{enumerate}

    In this example we would
    use a standard PLSA, other builders work analogously.  To build plsa one should
    \begin{enumerate}
	\item Set number of topics in model: \\ \texttt{ val numberOfTopics = 25}
	\item Set the number of iterations for EM\--algorithm to perform: \\ \texttt{ val numberOfIteration = 100}
	\item Create an instance of class java.util.Random: \\     \texttt{ val random = new Random()}
	\item And build the model: \\ \texttt{ val builder = new PLSABuilder(numberOfTopics, alphabet, documents, random,  numberOfIteration)}
    \end{enumerate}

\subsection{Training a model}
    Now we build a model and can perform stochastic matrix decomposition(train the model) \\
    \texttt{val trainedModel = plsa.train(documents)}\\
    trainedModel holds the matrices $\Phi$ and $\Theta$ (see \ref{matrixDecomposition})
    $\Phi$ is distribution of words by topic thus the number in the intersects of $i\--th$ row and $j\--th$ column show the
    probability to generate word $j$ from topic $i$. Each attribute set according to the matrix $\Phi$.
    To obtain matrix $\Phi$, corresponds to attribute Category:\\
    \texttt{val phi = trainedModel.phi(Category)}\\
    $\Theta$ is a distribution of documents by topic thus the number in the intersects of $i\--th$ row and $j\--th$ column show
    the weight of topic $j$ in document $i$.
    To obtain matrix $\Theta$: \\
    \texttt{val theta = trainedModel.theta}\\

    \subsubsection{How to interpret the results}
	If you are not familiar with topic modeling read \ref{generativeModel}. In our library we store the distribution of words by topic in matrix $\Phi$,
	distribution of documents by topics in matrix $\Theta$. In order to obtain the probability to generate word $w$ from topic $t$ one may use method \\
	\texttt{trainedModel.phi(Category).probability(t, w)}\\
	In order to obtain the probability to generate topic $t$ in document $d$ \\
	\texttt{trainedModel.theta.probability(d, t)}\\
    
    \subsubsection{How to deal with matrices}
	Matrices $\Phi$ and $\Theta$ hold different information, but support the similar methods.
	Both classes hold matrix of expectation and stochastic matrix. Expectation matrix holds values, estimated by E\--step (see \ref{EMAlgorithm}),
	stochastic matrix contain a probability. Method of classes Phi and Theta are explained in table \ref{phiAndThetaMethods}
	    
	\begin{table}[ht!]
	    \caption{Method of class Phi and class Theta}
	    \label{phiAndThetaMethods}
	    \begin{tabular}{|l|p{4cm}|p{4cm} |p{4cm} |}
		\hline
		method & Phi & Theta \\
		\hline
		    addToExpectation($row$, $column$, $value$)
		    & $row$ \-- topic index column \-- word index  $column$ \-- topic index
		    & $row$ \-- document index  value \-- $n_{dwt}$  value \-- $n_{dwt}$ \\
		\hline
		    probability($rowIndex$, $columnIndex$)
		    & probability to generate word $columnIndex$ from topic $rowIndex$
		    & weight of topic $columnIndex$ in document $rowIndex$\\
		\hline
		    expectation($rowIndex$, $columnIndex$) &
		    return expected value for words, $columnIndex$ generated from topic $rowIndex$: $n_{wt}$ &
		    return expected value for words,  generated from topic $columnIndex$ in document $rowIndex$: $n_{dt}$ \\
		\hline
		    numberOfRows & number of topics & number of documents \\
		\hline
		    numberOfColumns & number of words& number of topics\\
		\hline
		dump() &\multicolumn{3}{|p{8.430cm} |}{replace negative value in expectation matrix by zero, replace value in stochastic matrix
		by corresponding values from expectation matrix, normalize stochastic matrix and replace values in expectation matrix by zero} \\ % <---- inserted &
		\hline
	    \end{tabular}
	\end{table}  
	We also have an utility to save matrix in file: \\
	\texttt{TopicHelper.saveMatrix("path/to/file", matrix)}
	where matrix may be $\Phi$ or $\Theta$. 
	One also may write top $n$ words from each topic to estimate topic coherence with 
	\texttt{TopicHelper.printAllTopics(n, phi, alphabet)}



    
    



