\subsection{Regularizer}
    The description of theory are described in \ref{Regularizers}. Hear we would describe the detail of implementation of regularizer in our project and describe some standard regularizers,
    which are implemented in our library.
    \subsubsection{Implementation}
	One may find regularizers in ru.ispras.modis.regularizer. Any regularizer should inherit from class Regularizer. In order to implement his own regularizer user have to implement methods
	regularizePhi, regularizeTheta and apply.
	\paragraph{Implementation of regularizePhi\\}
	As one may see in \ref{RegularizersEquation},
	\begin{equation} \varphi_{wt} \propto \left(\hat{n}_{wt} + \varphi_{wt} \frac{\partial  R(\Phi, \Theta)}{\partial \varphi_{wt}} \right)_+ \end{equation}
	Thus, in order to implement method regularizePhi one have to:
	\begin{itemize}
	    \item calculate $\varphi_{wt} \frac{\partial  R(\Phi, \Theta)}{\partial \varphi_{wt}}$ using matrix $\Phi$ and matrix $\Theta$ for each word $w$ and
		each topic $t$.\\
		In order to take $i$\--th row and $j$\--th column in matrix $\Phi$ one may use $$phi.probability(i, j)$$ where $i$\--- topic index,
		$j$\--- word index.\\
		Analogously for matrix $\Theta$: $$theta.probability(i, j)$$, where $i$\--topic index, $j$ \-- document index.
	    \item Add these values to matrix of expectation, in order to add $\varphi_{wt} \frac{\partial  R(\Phi, \Theta)}{\partial \varphi_{wt}}$ to
		expectation matrix $n_{wt}$ one should use method\\
		$$phi.addToExpectation(t, w, \varphi_{wt} \frac{\partial  R(\Phi, \Theta)}{\partial \varphi_{wt}})$$
	\end{itemize}
	
    \paragraph{Implementation of regularizeTheta \\}
	This method is analogous to the previous paragraph. One have to calculate $$ \theta_{td}\frac{\partial  R(\Phi, \Theta)}{\partial \varphi_{td}} $$
	and add it to expectation of $n_{dt}$.
    \paragraph{Implementation of apply \\}
	apply is used to calculate \ref{optimizeWithReqularizer} instead of log likelihood (and corresponding perplexity). If you wont calculate log likelihood or
	you are lazy to implement this method return 0f. 

    \subsubsection{Implemented reqularizer}
	Now in our project are implemented following regularizers:
	\begin{itemize}
	    \item ZeroRegularizer, it regularizer do nothing, if you don't wont to use regularizer use this one
	    \item RegularizerSum allow to apply a sequence of regularizers sequentially. For example if you have a few reqularizer: $r_1$, $r_2$, $r_3$ and you wont to
		apply them sequentially. For this aim\\
		\texttt{import ru.ispras.modis.regularizer.Regularizer.toRegularizerSum}
		\texttt{val regularizerSum = $r_1$ + $r_2$ + $r_3$}
	    \item SymmetricDirichlet, it regularizer add a dirichlet prior to the distribution of document by topic and words by topic, it is used to convert PLSA into LDA (see \ref{LDA})
 	\end{itemize}

	